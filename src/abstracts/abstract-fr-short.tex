\section*{Résumé vulgarisé}%
\label{sec:pop-abstract}

La vision par ordinateur est la science de la compréhension de l'environnement à partir d'images,
très utile dans des domaines tels que la réalité augmentée ou les véhicules autonomes.

Dans un premier temps, nous nous attaquons problème de l'annotation pour la segmentation d'image.
Cette tâche consiste à identifier les pixels correspondants
à un objet d'intérêt dans une image pour entraîner un ordinateur à
le faire automatiquement plus tard.
Nous présentons notre approche pour annoter efficacement des grands ensembles d'images
par le biais d'un service de crowdsourcing en ligne.

Dans un second temps, nous étudions le problème d'odométrie visuelle,
qui consiste à retracer la trajectoire d'une caméra et à reconstruire
une carte virtuelle en 3D de son environnement à partir des images que la caméra fournit.
Nous présentons notre bibliothèque logicielle libre, ainsi qu'une application Web capable de
d'améliorer les résultats de l'algorithme grâce à des interactions utilisateur.
