\clearpage
\section*{Résumé}%
\label{sec:resume}

\textbf{Mots-clé : vision par ordinateur; interactif; Web; annotation; odometrie.}

La vision par ordinateur est un domaine de l'informatique visant à reproduire et à améliorer
la capacité de la vision humaine à comprendre son environnement.
Dans cette thèse, nous nous concentrons sur deux domaines de la vision par ordinateur,
à savoir la segmentation d'image et l'odométrie visuelle.
Nous montrons l'impact positif qu'apporte l'usage
d'applications Web interactives pour chacun d'eux.

La première partie de cette thèse porte sur l'annotation et la segmentation d'images.
Nous définissons dans un premier temps le problème de l'annotation d'images
et les défis que cela représente pour des grands ensembles de données.
De nombreuses interactions ont été utilisées dans la littérature
pour aider les algorithmes de segmentation.
Les plus courantes consistent à désigner explicitement des contours,
dessiner des boîtes englobantes,
ou marquer des traits à l'intérieur et à l'extérieur des objets d'intérêt.
Dans un contexte de crowdsourcing, les tâches d'annotation sont déléguées
à un public non-expert.
Pour cette raison, nous avons mené une étude utilisateur montrant
les avantages d'une interaction que nous appelons entourage par rapport aux autres types d'interactions.
Nous décrivons comment le langage de programmation Elm nous a aidé à
construire une application Web d'annotation d'images qui soit fiable.
Un tour d'horizon des fonctionnalités et de son architecture est proposé,
ainsi qu'un guide pour le déploiement dans des services de microtâches comme Amazon Mechanical Turk.
Cette application est entièrement libre et mise à disposition en ligne.

Dans la seconde partie de cette thèse,
nous présentons notre bibliothèque libre d'odométrie visuelle directe.
Nous fournissons une évaluation comparative montrant que notre approche
est aussi performante que les alternatives actuellement disponibles.
La formulation du problème d'odométrie visuelle
repose sur des outils géométriques et des techniques d'optimisation
nécessitant une grosse puissance de calcul pour fonctionner à 25 images par seconde.
Puisque nous aspirons à exécuter ces algorithmes sur le Web,
nous passons en revue les technologies passées et courantes
fournissant des bonnes performances directement au sein du navigateur Web.
En particulier, nous détaillons comment cibler une nouvelle plateforme appelée WebAssembly
à partir des langages de programmation C++ et Rust.
Notre bibliothèque a été implémentée entièrement dans le langage de programmation Rust,
ce qui en a facilité le portage vers WebAssembly.
Cette propriété nous a permis de mettre en place
une application Web d'odométrie visuelle proposant différents types d'interactions.
Une barre de temps permet une navigation unidimensionnelle le long de la séquence vidéo.
Des paires de points peuvent être sélectionnées sur deux images
de la séquence pour réaligner les caméras et corriger l'éventuelle dérive.
Des couleurs sont également utilisées pour identifier des parties sélectionnables
du nuage de points 3D pour réinitialiser les positions de la caméra.
La combinaison de ces interactions permet d'apporter des améliorations
sur les résultats du suivi et de la reconstruction du nuage de points 3D.
