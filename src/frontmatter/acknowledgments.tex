First I'd like to thank my advisors Vincent and Axel without whom
that PhD would not have been possible.
I would also like to thank Véronique and Mathias for reviewing this manuscript,
as well as the other members of the jury, Oge, Géraldine and Christophe for your attention,
remarks and interesting discussions during the defense.
Again, a special thank you Axel
for all that you've done, throughout this long period and even before it started.
I haven't been on the easiest path toward completion of this PhD but you've
always been there to help me continue being motivated and that's what mattered most!


Wanting to begin a PhD certainly isn't a one-time moment,
but for me, the feeling probably started during my M1 internship.
I was working in the VORTEX research team (now REVA)
on a project with Yvain and Jean-Denis and it was great!
Yet ``I don't think so'' was more or less what I kept answering to my teachers
when they would ask if I wished to start a PhD at that time.
And it lasted until almost the end of my M2 internship in Singapore,
when working on a project to transfer research to a product.
The internship and my first experience in Singapore was great and I'd like
to especially thank Axel, Vincent, Thanh, Nizar and Igor for that.
It was at that period I realized this is what I wanted!


As a consequence, a significant time of this PhD was spent in Singapore.
It is at the origin of all the visual odometry part of this thesis.
I'd like to thank Mounir, Xiong Wei and Janice in addition
to my supervisors who made that possible.
I also met amazing friends there
without whom this would have been a much different experience.
Justin, Yolanda, Thomas, Ariane, Flo, Bastien, Clovis, Martin, Ana and Joaquim thank you!


The major part of my PhD was spent in Toulouse,
surrounded by wonderful colleagues and friends.
Some of them became futsal mates, running partners,
Tarot and Belotte players, oxidizers (increasing my amount of Rust),
temporary flatmates,
gaming friends, and night watchers (on TV as well as in bars!).
In the lab, it's all the little things, from Super AdMinistrative powers
to team workshops or even Thursday burgers,
that add up to form a very welcoming and enjoying working environment.
Outside the lab, I've been lucky to meet all my friends in Toulouse
and I hope that I'll be able to keep in touch.
Jean-Denis, Yvain, Simone, Géraldine, Sylvie, Pierre, Charlie, Sam,
Axel, Vincent C, Vincent A, Thibault, Bastien, Chafik, Arthur, Julien, Paul,
Thomas, Matthieu, Thierry,
Damien, Sonia, Jean, Richard, Simon, Patrick, Alison, Etienne,
Antoine, Matthias, Nicolas, Korantin thank you all!
A dedication also to the friends that helped me becoming who I am
before joining the pink city, Yanis, Alexandre, Alain, Baptiste,
Bastien, Océane and Alice thank you!


Finally, I would like to thank my family for always encouraging me
and giving me the means to pursue that science quest of mine.
En particulier, merci Papa, Maman.
