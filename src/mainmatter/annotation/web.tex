\chapter{Reliable Web Applications}%
\label{cha:reliable_web_applications}

\begin{itemize}
	\item What is the Web?
	\begin{itemize}
		\item What is a Web application?
		\item Rich Internet Application (RIA)?
	\end{itemize}
	\item JavaScript or ECMAScript
	\begin{itemize}
		\item Creation of JavaScript
		\item Browser performance war with Just In Time compilation (JIT)
		\item Explosion of JavaScript and Node
		\item JavaScript issues
		\item The ``transpilation to JavaScript'' paradigm
	\end{itemize}
	\item Frontend Web programming
	\begin{itemize}
		\item Single Page Application (SPA)?
		\item Reactive programming
		\item Functional Reactive programming (FRP)
		\item About the Virtual DOM
		\item Async and the event loop
	\end{itemize}
	\item Elm
	\begin{itemize}
		\item Pure functions
		\item Algebraic Data Types (making impossible states impossible)
		\item Totality
		\item Switch to The Elm Architecture (TEA)
		\item 0 runtime exception
		\item Elm-UI, an alternative layout approach
	\end{itemize}
\end{itemize}


\section{What is the Web?}%
\label{sec:web}

The Internet and the Web are ubiquitous technologies of our everyday lives.
Social media, communication, search, news, entertainment, mapping,
shopping, learning, allmost every activity is now digital and online.
Simply put, the Web, also called World Wide Web (WWW), consists of the sum of all resources,
available through unique identifiers (URI), that we share on the Internet,
the global network carrying them.

In this chapter, we will recap the Web main evolutions,
from static content to dynamic applications,
and explain the choices we made to build reliable annotation Web applications.

\subsection{What is a Web application?}%
\label{sub:web_application}

An application, in the context of programming (/computers),
is of a piece of software presenting information to a user,
usually in an actionable manner.
This includes things like email clients, image manipulation, video games,
word processors, automatic translation, and virtually any functionality
available with a regular computing device.

Web resources are commonly accessible through a Web browser.
As of May 2019, the three most used Web browsers are Google Chrome (62.7\%),
Apple Safari (15.9\%) and Mozilla Firefox (5.1\%).
Cf http://gs.statcounter.com/browser-market-share/all/worldwide/2019.
Thus, we can define a Web application as a user-facing software,
accessed through a Web browser.


\subsection{Rich Web Application}%
\label{sub:rich_web_application}

Traditionally, websites used to present their resources in the form of a collection
of static documents, known as Web pages, linked together with hyperlinks.
The nature of web pages would mostly be informative, visual or textual,
with very few other interactions than navigation through the site by
clicking on the links.
