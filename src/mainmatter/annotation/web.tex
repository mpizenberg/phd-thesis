\chapter{Reliable Web Applications}%
\label{cha:reliable_web_applications}

\begin{itemize}
	\item What is the Web?
	\begin{itemize}
		\item What is a Web application?
		\item Rich Internet Application (RIA)?
	\end{itemize}
	\item JavaScript or ECMAScript
	\begin{itemize}
		\item Creation of JavaScript
		\item Browser performance war with Just In Time compilation (JIT)
		\item Explosion of JavaScript and Node
		\item JavaScript issues
		\item The ``transpilation to JavaScript'' paradigm
	\end{itemize}
	\item Frontend Web programming
	\begin{itemize}
		\item Single Page Application (SPA)?
		\item Reactive programming
		\item Functional Reactive programming (FRP)
		\item About the Virtual DOM
		\item Async and the event loop
	\end{itemize}
	\item Elm
	\begin{itemize}
		\item Pure functions
		\item Algebraic Data Types (making impossible states impossible)
		\item Totality
		\item Switch to The Elm Architecture (TEA)
		\item 0 runtime exception
		\item Elm-UI, an alternative layout approach
	\end{itemize}
\end{itemize}


\section{What is the Web?}%
\label{sec:web}

The Internet and the Web are ubiquitous technologies of our everyday lives,
created around the 80's.
Social media, communication, search, news, entertainment, mapping,
shopping, learning, allmost every activity is now digital and online.
Simply put, the Web, also called World Wide Web (WWW), consists of the sum of all resources,
available through unique identifiers (URI), that we share on the Internet,
the global network carrying them.

In this chapter, we will recap the Web main evolutions,
from static content to dynamic applications,
and explain the choices we made to build reliable annotation Web applications.

\subsection{What is a Web application?}%
\label{sub:web_application}

An application, in the context of programming (/computers),
is a piece of software presenting information to a user,
usually in an actionable manner.
This includes things like email clients, image manipulation, video games,
word processors, automatic translation, and virtually any functionality
available with a regular computing device.

Web resources are commonly accessible through a Web browser.
As of May 2019, the three most used Web browsers are Google Chrome (62.7\% of global market share),
Apple Safari (15.9\%) and Mozilla Firefox (5.1\%).
Cf http://gs.statcounter.com/browser-market-share/all/worldwide/2019.
Thus, we can define a Web application as a user-facing software,
accessed through a Web browser.

The three pillars of Web applications are HTML, CSS and JavaScript.
HTML, for ``Hypertext Markup Language'' is a description language
orginizing a page information as a hierarchy of tagged content.
In Listing~\ref{lst:html}, a ``body'' tag contains three other tags,
a title ``h1'' (h for header), a paragraph ``p'', an image ``img''
and a button not yet linked to any action.
CSS, for ``Cascading Style Sheet'', complements HTML by styling
the content of associated HTML documents.
Listing~\ref{lst:css} shows how we would add a left margin of 20 pixels
on all the document body, and make the h1 title red and bold.
JavaScript is a scripting language, not affiliated in any form
to the Java programming language.
It is run inside the browser to add dynamic behavior to a Web page.
In Listing~\ref{lst:js} we show how one could count and display
the number of times a user clicked on the button in the page.

\lstset{style=CodeStyle}
\lstinputlisting[language=html,caption={Example HTML code.},label={lst:html}]{assets/code/html.html}
\lstinputlisting[language=css,caption={Example CSS code.},label={lst:css}]{assets/code/css.css}
\lstinputlisting[language=js,caption={Example JavaScript code.},label={lst:js}]{assets/code/js.js}

\subsection{Rich Web Application}%
\label{sub:rich_web_application}

Traditionally, websites used to present their resources in the form of a collection
of static documents, known as Web pages, linked together with hyperlinks.
The nature of web pages would mostly be informative, visual or textual,
with very few other interactions than navigation through the site by
clicking on the links.

Today, thanks to evolutions of Web technologies that we will detail later,
Web applications have become full-fledged applications with almost
the same capabilities as desktop ones.
They feature functionalities like 3D graphics, sound processing or interactive elements,
and are sometimes called rich web applications.
Similar concepts like ``progressive web applications'' (PWA),
or ``single page applications'' (SPA) are also explained in the following sections.
Now let's dive into the cornerstone of Web pages dynamism, JavaScript.


\section{JavaScript, formally known as ECMAScript}%
\label{sec:javascript_formally_known_as_ecmascript}

\subsection{Genesis of JavaScript}%
\label{sub:genesis_of_javascript}

In 1995, the dominating Web browser was the Netscape Navigator.
Realizing that pages dynamism was key in the war against Microsoft
own Web technologies, Netscape Communications recruited Brendan Eich,
with the aim of integrating a scripting language into their browser.
And so, in May 1995, he wrote a prototype in 10 days.
Assumably for marketing reasons, it was officially named JavaScript
when realeased in Netscape Navigator 2.0 beta 3.

Two years later, in June 1997, the European Computer Manufacturers Association
(ECMA) standardized the first version of ``ECMAScript'' as ECMA-262,
JavaScript being its most well known implementation.
The ECMAScript (ES) standard has been evolving since.
Today, all browsers fully implement ES5, released in 2009,
and partially implement the most recent versions, ES2015,
ES2016, ES2017 and ES2018.

\subsection{Browser performance war}%
\label{sub:browser_performance_war}

Though many browser wars for dominance of market share occurred since the 90's,
we will focus on the JavaScript engine performance war, starting around 2008
when Google released Chrome.
