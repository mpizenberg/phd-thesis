\chapter{Performant Web Applications}%
\label{cha:performant_web_applications}

\minitoc%

\section{A Brief History of Native Code in the Client}%
\label{sec:native_client}

History of native code on the Web: Java, Flash, nacl, wasm.
Flash, ActionScript programming language.

------------------------- Java Applet

Appeared in 1995 with the first version of Java.
Unlike JS, Java applets had access to 3D hardware acceleration,
so used for computer intensive tasks and visualizations.
(Until HTML5, canvas and webgl)

Break accessibility. No screen readers.
No Java applets in mobile browsers.
Security relies on users approving ("trust") an applet.


In 2013, browsers deprecate NPAPI (Netscape Plugin API).
In 2018, September, Java applets are removed from Java SE 11.


------------------------- Flash

https://web.archive.org/web/20080120200126/http://www.adobe.com:80/macromedia/events/john_gay/index.html

In 1993, Jonathan Gay, Charlie Jackson, Michelle Welsh founded FutureWave Software.
In 1995, after discussions at Siggraph,
the company decided to focus on web animation with product "FutureSplash Animator".
The Simpsons website in 1996 was done with FutureSplash.
They gained reputation with Microsoft and Disney and sold the company to Macromedia.
It becamed Macromedia Flash 1.0.
In 2001, 50 people working on it. and Flash Player was the most ubiquitous software on the internet.

https://www.apple.com/hotnews/thoughts-on-flash/
Thoughts on Flash, Steve Jobs, April 2010.
 - Flash is proprietary
 - Flash is the number one reason Macs crash
 - Flash has security flaws
 - Flash uses more energy than hardware accelerated H264 video in HTML5
 - Flash is not adapted to touch interactions
 - Flash hinders platform improvements on Apple products

https://www.cvedetails.com/product/6761/Adobe-Flash-Player.html?vendor_id=53
In 2015, 329 CVEs, 283 leading to arbitrary code execution.

In 2000, Flash 5 with ActionScript was released.
In 2002, Flash 6, video support.
In 2005, Chad Hurley, Steve Chen and Jawed Karim launched YouTube, with Flash.
In 2007, iPhone, no Flash.

https://www.nngroup.com/articles/flash-99-percent-bad/
Usability consultant Jakob Nielsen published an Alertbox in 2000 entitled,
Flash: 99\% Bad,
stating that "Flash tends to degrade websites for three reasons:
 - it encourages design abuse,
 - it breaks with the Web's fundamental interaction principles,
 - and it distracts attention from the site's core value."

In July 2017, Adobe announced end support for Flash in 2020.

------------------------- NaCl

Native Client (NaCl) project in chrome
A lot of C and C++ code already written.
Secure execution of native code.
-> Changing the compiler to produce only "secure" code, inside a sandbox.
-> no plugin "please install"
-> check at runtime if the application respect the safety rules, or shutdown.

Plugins have historically been the number one source of web browser vulnerabilities.

Pepper API, mirror of the Browser API, for NaCl code.
built-in flash using Pepper and Pdf reader using Pepper.

NaCl is architecture specific. Started with the GCC toolchain.
Need to go through the Chrome web store.

PNaCl, an intermediate representation, compile on the fly. Uses Clang.
Based on the LLVM toolchain.

Focused on desktop, no mobile for now.

------------------------

Java and Flash are technologies owned by companies, this does not integrate well with the Web.
Asm.js was just JS. "Don't break the web".
WebAssembly is a W3C spec, it went through the process, not against it.
Wasm is also small. Flash and Java Applets needed to download a runtime. and PNaCl was built
on top of LLVM IR, a bit unstable.
Wasm does not force you to use a specific language like Flash or Java.

Wasm has security in mind, it runs in the same sandbox than JS.

\section{Emscripten}%
\label{sec:Emscripten}

\begin{itemize}
	\item LLVM
	\item asmJS
	\item wasm
\end{itemize}

\section{WebAssembly}%
\label{sec:WebAssembly}

\begin{itemize}
	\item Minimum Viable Product (MVP)
	\item Bright future: wasi, wapm, ...
\end{itemize}

\section{C++ portability pitfalls}%
\label{sec:cpp_pitfalls}

\section{Rust and WebAssembly}%
\label{sec:rust_wasm}

\begin{itemize}
	\item Rust programming language
	\item Wasm-bindgen
\end{itemize}
