\chapter{Performant Web Applications}%
\label{cha:performant_web_applications}

\minitoc%

\section{A Brief History of Native Code in the Client}%
\label{sec:native_client}

History of native code on the Web: Java, Flash, nacl, wasm.
Flash, ActionScript programming language.

------------------------- Java Applet

Appeared in 1995 with the first version of Java.
Unlike JS, Java applets had access to 3D hardware acceleration,
so used for computer intensive tasks and visualizations.
(Until HTML5, canvas and webgl)

Break accessibility. No screen readers.
No Java applets in mobile browsers.
Security relies on users approving ("trust") an applet.


In 2013, browsers deprecate NPAPI (Netscape Plugin API).
In 2018, September, Java applets are removed from Java SE 11.


------------------------- Flash

https://web.archive.org/web/20080120200126/http://www.adobe.com:80/macromedia/events/john\_gay/index.html

In 1993, Jonathan Gay, Charlie Jackson, Michelle Welsh founded FutureWave Software.
In 1995, after discussions at Siggraph,
the company decided to focus on web animation with product "FutureSplash Animator".
The Simpsons website in 1996 was done with FutureSplash.
They gained reputation with Microsoft and Disney and sold the company to Macromedia.
It becamed Macromedia Flash 1.0.
In 2001, 50 people working on it. and Flash Player was the most ubiquitous software on the internet.

https://www.apple.com/hotnews/thoughts-on-flash/
Thoughts on Flash, Steve Jobs, April 2010.
 - Flash is proprietary
 - Flash is the number one reason Macs crash
 - Flash has security flaws
 - Flash uses more energy than hardware accelerated H264 video in HTML5
 - Flash is not adapted to touch interactions
 - Flash hinders platform improvements on Apple products

https://www.cvedetails.com/product/6761/Adobe-Flash-Player.html?vendor\_id=53
In 2015, 329 CVEs, 283 leading to arbitrary code execution.

In 2000, Flash 5 with ActionScript was released.
In 2002, Flash 6, video support.
In 2005, Chad Hurley, Steve Chen and Jawed Karim launched YouTube, with Flash.
In 2007, iPhone, no Flash.

https://www.nngroup.com/articles/flash-99-percent-bad/
Usability consultant Jakob Nielsen published an Alertbox in 2000 entitled,
Flash: 99\% Bad,
stating that "Flash tends to degrade websites for three reasons:
 - it encourages design abuse,
 - it breaks with the Web's fundamental interaction principles,
 - and it distracts attention from the site's core value."

In July 2017, Adobe announced end support for Flash in 2020.

------------------------- NaCl

Native Client (NaCl) project in chrome
A lot of C and C++ code already written.
Secure execution of native code.
-> Changing the compiler to produce only "secure" code, inside a sandbox.
-> no plugin "please install"
-> check at runtime if the application respect the safety rules, or shutdown.

Plugins have historically been the number one source of web browser vulnerabilities.

Pepper API, mirror of the Browser API, for NaCl code.
built-in flash using Pepper and Pdf reader using Pepper.

NaCl is architecture specific. Started with the GCC toolchain.
Need to go through the Chrome web store.

PNaCl, an intermediate representation, compile on the fly. Uses Clang.
Based on the LLVM toolchain.

Focused on desktop, no mobile for now.

------------------------

Java and Flash are technologies owned by companies, this does not integrate well with the Web.
Asm.js was just JS. "Don't break the web".
WebAssembly is a W3C spec, it went through the process, not against it.
Wasm is also small. Flash and Java Applets needed to download a runtime. and PNaCl was built
on top of LLVM IR, a bit unstable.
Wasm does not force you to use a specific language like Flash or Java.

Wasm has security in mind, it runs in the same sandbox than JS.

\section{Emscripten}%
\label{sec:Emscripten}

Faisons un bref détour "historique". Alon Zakai, qui travaille à Mozilla, a démarré un projet perso autour de 2010 par curiosité (vidéo, autre vidéo). Il se demandait s’il était possible de transformer du code C++ en du code JavaScript pour le faire tourner dans un navigateur. C’est le début de emscripten. Emscripten transforme du code LLVM en JavaScript. LLVM est ce qu’on appelle une représentation intermédiaire (IR) qui permet à du code humain (C++, rust, ...) d’être transformé en code LLVM qui lui même est après transformé en code spécifique à une architecture (32 bits, 64 bits x86 intel, ARM, embarqué, ...).

Et donc ce que Alon Zakai a démarré en 2010 c’est une nouvelle "cible" : JavaScript. Initialement c’était assez lent, loin d’être optimisé. Après quelques itérations, avec l’aide de Luke Wagner qui travaillait sur le compilateur JavaScript, ils ont formalisé une syntaxe spécifique. Un sous-ensemble de JavaScript qui peut être facilement reconnu par le compilateur du navigateur et optimisé bien plus efficacement. Cette syntaxe ressemblait à ça

function sum(x, y) {
    var x = x | 0;  // x is a 32-bit value
    var y = y | 0;  // so is y
    return (x+y) | 0; // 32-bit addition, no type or overflow checks
}

Normalement, lorsque l’on fait x + y en JavaScript, le compilateur doit vérifier les types des deux variables dynamiquement choisir la bonne opérations somme en conséquence. Le faire au runtime est très coûteux. Alors qu’avec cette syntaxe, formalisée sous le nom de asm.js, le compilateur peut directement émettre l’instruction processeur de somme de deux entiers 32 bits. Après avoir intégré la techno avec des équipes de Unity et Unreal pour faire tourner leur moteur de jeu dans le navigateur, la preuve de concept était faite. Il s’en est suivi un peu de diplomatie entre les gros du web (Google, Microsoft, Mozilla) pour formaliser WebAssembly, le successeur de asm.js.

Ce petit détour permet de comprendre ce qu’est emscripten, comment le projet à évolué jusqu’à aujourd’hui où on peut l’utiliser pour compiler du code C/C++ vers wasm. La clé de la compréhension c’est ce schéma :

Code source (C++) -> LLVM IR -> Emscripten -> WebAssembly

Donc Emscripten a besoin du bitcode LLVM pour pouvoir compiler une base de code C++. En particulier, il n’est pas possible d’utiliser les bibliothèques précompilées (les “.so”) fournies par les OS dans les packages (apt-get install ...) dans ses dépendances. Toute dépendance (et ce récursivement) doit être recompilée avec Emscripten depuis le code source, et liée statiquement à la compilation de notre propre code. Par défaut, pour faciliter la vie, Emscripten intègre déjà la majorité de la librairie standard (std), et de certaines autres bibliothèques (SDL pour les applis graphiques, ...).


\begin{itemize}
	\item LLVM
	\item asmJS
	\item wasm
\end{itemize}

\section{WebAssembly}%
\label{sec:WebAssembly}

\begin{itemize}
	\item Minimum Viable Product (MVP)
	\item Bright future: wasi, wapm, ...
\end{itemize}

\section{C++ portability pitfalls}%
\label{sec:cpp_pitfalls}

Example with image reading in OpenCV.
Had to rewrite PNG decoder.


DVO par exemple a des dépendances à OpenCV, Eigen, Boost, et Sophus. J’ai donc commencé par voir si j’arrive à compiler un programme minimaliste OpenCV avec Emscripten. Tout de suite les ennuis ont démarré. Au début, je n’ai même pas réussi à utiliser le build wasm de OpenCV (cf question sur stackoverflow). Je m’en suis finalement sorti, pour bloquer sur l’étape d’après, lire une image (cf question sur le forum OpenCV). En effet, dans la version wasm de OpenCV, la fonction imread a été remplacée par une méthode purement JavaScript qui prend un canvas en paramètre, et va donc directement récupérer la matrice depuis le canvas sans faire le décodage. Je pense que cette stratégie a été utilisée parce qu’en interne, la fonction imread OpenCV utilise la fonction imdecode, qui elle-même fait appel à une douzaine d’autres bibliothèques (libjpeg, libtiff, libpng, ...). J’ai essayé d’ajouter moi-même le module contenant la fonction imdecode au build wasm d’OpenCV mais je n’ai pas réussi à compiler ce module. J’ai tout un tas d’erreurs de "missing symbol" etc.

Ce n’est pas entièrement sans espoir, il faut le savoir ! Notamment, je n’ai besoin que du décodage png. Et libpng fait partie de ces quelques lib déjà intégrées à emscripten. En théorie je pourrais donc remplacer le code de décodage (cv::imdecode) par l’utilisation de libpng directement. Un des soucis c’est que l’API de libpng est super difficile à comprendre pour moi. Mais en théorie, le reste d’OpenCV qui est utilisé dans DVO fait partie du sous-ensemble qui devrait marcher.

Par contre, je n’ai pas fait de tests minimalistes avec les autres dépendances de DVO (eigen, boost, sophus) donc rien ne dit qu’on serait au bout de nos ennuis.

\section{Rust and WebAssembly}%
\label{sec:rust_wasm}

\begin{itemize}
	\item Rust programming language
	\item Wasm-bindgen
\end{itemize}
